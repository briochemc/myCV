%!TEX TS-program = lualatex
\documentclass[12pt]{friggeri-cv}
\addbibresource{mypapers.bib}

\definecolor{darker_blue}{HTML}{4063D8}

\usepackage{hyperref}
\hypersetup{
    colorlinks=true,       % no lik border color
    allcolors=darker_blue,         % color of internal links (change box color with linkbordercolor)
    allbordercolors=white  % white border color for all
}
%\addbibresource{bibliography.bib}

\definecolor{pblue}{RGB}{0,0,0}

% References 2 columns
\newcommand{\cvRefsTwoCols}[2]{
  \begin{minipage}[t]{.45\textwidth}#1\end{minipage}
  \hfill
  \begin{minipage}[t]{.45\textwidth}#2\end{minipage}
}

% Skills 2 columns
\newcommand{\cvSkillsTwoCols}[2]{
  \begin{minipage}[t]{.5\textwidth}#1\end{minipage}
  \hfill
  \begin{minipage}[t]{.4\textwidth}#2\end{minipage}
}

\newcommand{\cvreference}[4]{
    {\headingfont\color{headercolor}#1}\\
    {#2}
    \begin{flushright}
    {\href{mailto:#3@#4}{\textbf{#3}@#4}\\}
    \end{flushright}
}




% Remove asterisk in bibliography
\usepackage{etoolbox}
\patchcmd{\thebibliography}{\section*{\refname}}{}{}{}


% tikz star rating
%\newcommand\score[2]{
%\pgfmathsetmacro\pgfxa{#1+1}
%\tikzstyle{scorestars}=[star, star points=5, star point ratio=2.25, inner sep=0.15em,anchor=outer point 3]
%\begin{tikzpicture}[baseline]
%  \foreach \i in {1,...,#2} {
%    \pgfmathparse{(\i<=#1?"lightgray":"white")}
%    \edef\starcolor{\pgfmathresult}
%    \draw (\i*1em,0) node[name=star\i,scorestars,fill=\starcolor]  {};
%   }
%   \pgfmathparse{(#1>int(#1)?int(#1+1):0}
%   \let\partstar=\pgfmathresult
%   \ifnum\partstar>0
%     \pgfmathsetmacro\starpart{#1-(int(#1))}
%     \path [clip] ($(star\partstar.outer point 3)!(star\partstar.outer point 2)!(star\partstar.outer point 4)$) rectangle
%    ($(star\partstar.outer point 2 |- star\partstar.outer point 1)!\starpart!(star\partstar.outer point 1 -| star\partstar.outer point 5)$);
%     \fill (\partstar*1em,0) node[scorestars,fill=lightgray]  {};
%   \fi
%
%,\end{tikzpicture}
%}

\begin{document}
\header{Beno\^{i}t~}{Pasquier}

\section{Affiliation}


Department of Earth System Science \hfill
  \faEnvelope~\href{mailto:pasquieb@uci.edu}{\textbf{pasquieb}@uci.edu} \\
University of California, Irvine \hfill
  \faEnvelope~\href{mailto:briochemc@gmail.com}{\textbf{briochemc}@gmail.com} \\
California, CA, 92697 \hfill
\faMobile~+61 477 859 021 / \faSkype~briochemc\\
USA \hfill
  \faGithub~\href{https://github.com/briochemc}{briochemc}
    / \faGitlab~\href{https://gitlab.com/benoitpasquier}{benoitpasquier} \\
\phantom{website:} \hfill \faGlobe~\href{https://www.bpasquier.com}{www.bpasquier.com}



%-------------------------------------
\section{Research Interests}
%-------------------------------------
A fascinating consequence of the fluid nature of open-ocean ecosystems and nutrient cycles is that perturbations in one part of the ocean can influence biological production on the other side of the world.
My focus has been on this interplay between the ocean's circulation and biology on the global scale.

During my PhD, I investigated fundamental scientific questions of global ocean biogeochemical cycles using cutting-edge mathematical tools.
Specifically, I explored the teleconnections of the global biological pump by developing and using state-of-the-art inverse models of the phosphorus, silicon, and iron cycles.
More recently, through my postdoctoral appointment, I have developed novel Green-function-based diagnostics to investigate the marine iron cycle with more detail than ever before.
Lately, I have been developing the Algebraic Implicit Biogeochemical Elemental Cycling System (the AIBECS), a Julia package to provide an easy API to create global marine biogeochemistry models in just a few commands, which I believe could become both a great teaching medium and the ideal research tool for, e.g., offline parameter optimization.

There is a need to improve the current representation of biogeochemical processes in models of the ocean.
There is also ample room to develop new tools that are simultaneously simple to use and understand, efficient and fast to run, and suitable for the novel diagnostics I have become familiar with.
By providing clear quantitative answers, these tools help decipher complex global interactions of the oceanic nutrient cycles.

I am also an advocate for scientific openness, to facilitate collaboration, code review, and reproducibility.







\newpage
%-------------------------------------
\section{Education}
%-------------------------------------
\begin{entrylist}
  \entry
    {2013\textemdash2017}
    {PhD in Applied Mathematics}
    {University of New South Wales, Sydney, Australia}
    {Supervisor: \hyperref[MH]{\textbf{Mark Holzer}}.
    Modeling and diagnosing ocean biogeochemical cycles.\\
    \textbf{Thesis title}: \emph{The Ocean's Global Iron, Phosphorus, and Silicon Cycles: Inverse Modelling and Novel Diagnostics}.
    \begin{itemize}
        \item Global Biogeochemical Cycles, Global Biological Pump
        \item Ecosystem Modeling \& Biogenic Transport Modeling
        \item Green Functions Techniques (Path Densities, Flow Rates, Time Scales)
        \item Inverse Modeling (Newton's Method for Root Finding and Optimization)
        \item Iron Control on the Global Biological Pump
        \item Southern Ocean Nutrient Trapping
    \end{itemize}
    }

  \entry
    {2010}
    {MSc in Environmental Science}
    {University of New South Wales, Sydney, Australia}
    {Study of the nature of environmental problems and the methodology of their evaluation and management.
    \begin{itemize}
      \item Geophysical Fluid Dynamics (taught by \hyperref[MH]{\textbf{Mark Holzer}})
      \item Oceanography (\hyperref[KM]{\textbf{Katrin Meissner}})
      \item Project Management, Environmental Risk Management
    \end{itemize}
    }

  \entry
    {2007\textemdash2008}
    {MSc in Finance Mathematics}
    {Paris Dauphine + ENSAE ParisTech, Paris, France}
    {MASEF (Mathematics of Insurance, Economics and Finance), Finance specialty.
    \begin{itemize}
        \item Stochastic Calculus, Levy Processes with Jumps
        \item Stochastic Differential Equations
        \item Numerical Methods (Monte Carlo)
    \end{itemize}
    }

  \entry
    {2004\textemdash2007}
    {MSc in Mathematics \& Engineering}
    {\'{E}cole Polytechnique, Palaiseau, France}
    {Pure mathematics specialization.
    \begin{itemize}
        \item Algebra, Arithmetics, Numerical Methods
        \item Differential Topology, Relativity
        \item Physics, Biology
    \end{itemize}
    }

  \entry
    {2001\textemdash2004}
    {Preparatory Classes}
    {Lyc\'{e}e Mass\'{e}na, Nice, France}
    {French Preparatory Classes, mathematics specialty.
    \begin{itemize}
        \item Linear Algebra, Topology, Numerical Methods
        \item Mechanics, Electromagnetism, Thermodynamics
    \end{itemize}
    }
\end{entrylist}







\newpage
%-------------------------------------
\section{Professional {\color{pblue}Exp}erience}
%-------------------------------------
\begin{entrylist}
  \entry
    {Sep$\,$17\textemdash{}Present}
    {Postdoctoral Research Scholar}
    {University of California, Irvine, CA, USA}
    {Working on developing new tools and on improving global biogeochemistry models with \hyperref[JKM]{\textbf{J.\,Keith Moore}} and \hyperref[FP]{\textbf{Fran\c{c}ois Primeau}}.}

  \entry
    {Mar$\,$17\textemdash{}Aug$\,$17}
    {Casual Research Assistant}
    {University of New South Wales, Sydney, Australia}
    {Continuing PhD work with \hyperref[MH]{\textbf{Mark Holzer}}.
    }

  \entry
    {Jun$\,$16\textemdash{}Dec$\,$16}
    {Mathematics Tutor}
    {University of New South Wales, Sydney, Australia}
    {\emph{Numerical Methods and Statistics}, 2nd year.
    }

  \entry
    {May$\,$11\textemdash{}Aug$\,$12}
    {Proposal Engineer}
    {Degr\'{e}mont, Suez Environnement, Sydney, Australia}
    {Managed tendering projects for Design, Construction, Maintenance and Operation contracts.
    Participated in business development, liaising with potential clients, advertising on company capabilities.
    }

  \entry
    {Jul$\,$08\textemdash{}Jun$\,$09}
    {Currency Trader Assistant}
    {Soci\'{e}t\'{e} G\'{e}n\'{e}rale Investment Banking, Paris, France}
    {MASEF Internship, researched new detection and calculation techniques for high frequency data used in automated arbitrage.
    In particular, developed algorithms to evaluate unbiased stochastic moments in real-time.
    }

  \entry
    {Apr$\,$07\textemdash{}Jul$\,$07}
    {Mathematics Research Intern}
    {\'{E}cole Polytechnique, Palaiseau, France}
    {\'{E}cole Polytechnique Specialty (Mathematics) Internship at the Laurent Schwartz Mathematics Center under the direction of \textbf{Jean Lannes}. Calculated the Witt ring of quadratic forms defined on number fields, on the field of $p$-adic numbers, and on Dedekind rings such as the integers.
    }

  \entry
  {Sep$\,$04\textemdash{}Feb$\,$05}
    {IT Intern}
    {Bioforce, Lyon, France}
    {Bioforce provides training and careers advice in aid programmes and logistics. Developed an Access database to improve communication and management.
    }
\end{entrylist}








%-------------------------------------
\section{Other Skills}
%-------------------------------------
\cvSkillsTwoCols{
\subsection{Scientific Programming}
\vspace{5pt}
\begin{tabular}{rl}
     {\thinfont\color{headercolor}{MATLAB / Julia}} & {Extensive use}\\
     {\thinfont\color{headercolor}{java / C++}} & {Competent}\\
     {\thinfont\color{headercolor}{Ruby / Python / Stan}} & {Little experience}\\
\end{tabular}
}{
\subsection{Languages}
\vspace{5pt}
\begin{tabular}{rl}
     {\thinfont\color{headercolor}{French}} & {First language}  \\
     {\thinfont\color{headercolor}{English}} & {Fluent}\\
     {\thinfont\color{headercolor}{Italian}} & {Intermediate}  \\
     {\thinfont\color{headercolor}{Japanese}} & {Novice}\\
\end{tabular}
}













\newpage
%-------------------------------------
\printbibsection{article}{\color{pblue}{Pub}lications}
%-------------------------------------





\newpage
%-------------------------------------
\printbibsection{inproceedings}{{\color{pblue}Talks and Posters}}
%-------------------------------------








\newpage
%-------------------------------------
\section{Honors and Awards}
%-------------------------------------
\begin{entrylist}

  \entry{2015}{Scolarship}{Cuomo Foundation, Monaco}
  {}

  \entry{2014}{Scolarship}{Fr\`{e}res Louis et Max Principale Foundation, Monaco}
  {}

    \entry{2014 - 2016}{Scolarship}{Monaco Government, Monaco}
  {Higher studies scholarship}

  \entry{2013}{Scolarship}{Monaco Government, Monaco}
  {H.S.H.~ The Prince Albert II Exceptional Scholarship}

  \entry{2013 - 2016}{Scolarship}{Monaco Scientific Centre, Monaco}
  {}

  \entry{2013 - 2016}{Tuition Fee Scholarship}{Graduate Research Shcool, UNSW, Sydney, Australia}
  {}

  \entry{2004 - 2008}{Scholarship}{Monaco Government, Monaco}{Higher studies scholarship}

\end{entrylist}







%-------------------------------------
\section{References}
%-------------------------------------
\cvRefsTwoCols{
  % First column
  \phantomsection \label{FP}
  \cvreference
    {Fran\c{c}ois Primeau}
    {Department of Earth System Science\\
     University of California, Irvine\\
     CA, 92697, USA}
    {fprimeau}
    {uci.edu}
  \phantomsection \label{MH}
  \cvreference
    {Mark Holzer}
    {Department of Applied Mathematics\\
     School of Mathematics and Statistics\\
     University of New South Wales\\
     NSW, 2035, Australia}
    {mholzer}
    {unsw.edu.au}
}
{% Second column
  \phantomsection \label{JKM}
  \cvreference
    {J.\,Keith Moore}
    {Department of Earth System Science\\
     University of California, Irvine\\
     CA, 92697, USA}
    {jkmoore}
    {uci.edu}
  \phantomsection \label{AM}
  \cvreference
    {Adam Martiny}
    {Department of Earth System Science\\
     University of California, Irvine\\
     CA, 92697, USA}
    {amartiny}
    {uci.edu}
}
\vspace{10pt}





\end{document}

%!TEX TS-program = lualatex
\documentclass[]{friggeri-cv}
\usepackage{hyperref}
\usepackage{color}

\hypersetup{
    colorlinks=false,       % no lik border color
    allbordercolors=white    % white border color for all
}
%\addbibresource{bibliography.bib}

\definecolor{pblue}{RGB}{0,0,0}

% References 3 columns
\newcommand{\cvRefsThreeCols}[3]{
  \begin{minipage}[t]{.3\textwidth}#1\end{minipage}
  \hfill
  \begin{minipage}[t]{.3\textwidth}#2\end{minipage}
  \hfill
  \begin{minipage}[t]{.3\textwidth}#3\end{minipage}
}

% Skills 2 columns
\newcommand{\cvSkillsTwoCols}[2]{
  \begin{minipage}[t]{.5\textwidth}#1\end{minipage}
  \hfill
  \begin{minipage}[t]{.4\textwidth}#2\end{minipage}
}

\newcommand{\cvreference}[5]{
    {\headingfont\color{headercolor}#1}\\
    {#2}
    \begin{flushright}
    {\faEnvelope~\href{mailto:#3@#4}{\textbf{#3}@#4}\\}
    {\faMobile~#5\\}
    \end{flushright}
}


% tikz star rating
%\newcommand\score[2]{
%\pgfmathsetmacro\pgfxa{#1+1}
%\tikzstyle{scorestars}=[star, star points=5, star point ratio=2.25, inner sep=0.15em,anchor=outer point 3]
%\begin{tikzpicture}[baseline]
%  \foreach \i in {1,...,#2} {
%    \pgfmathparse{(\i<=#1?"lightgray":"white")}
%    \edef\starcolor{\pgfmathresult}
%    \draw (\i*1em,0) node[name=star\i,scorestars,fill=\starcolor]  {};
%   }
%   \pgfmathparse{(#1>int(#1)?int(#1+1):0}
%   \let\partstar=\pgfmathresult
%   \ifnum\partstar>0
%     \pgfmathsetmacro\starpart{#1-(int(#1))}
%     \path [clip] ($(star\partstar.outer point 3)!(star\partstar.outer point 2)!(star\partstar.outer point 4)$) rectangle
%    ($(star\partstar.outer point 2 |- star\partstar.outer point 1)!\starpart!(star\partstar.outer point 1 -| star\partstar.outer point 5)$);
%     \fill (\partstar*1em,0) node[scorestars,fill=lightgray]  {};
%   \fi
%
%,\end{tikzpicture}
%}

\begin{document}
\header{Beno\^{i}t~}{Pasquier}

\section{Affiliation}


    Department of Earth System Science \hfill
    \faEnvelope~\href{mailto:pasquieb@uci.edu}{\textbf{pasquieb}@uci.edu} \\
    University of California, Irvine \hfill
    \faEnvelope~\href{mailto:briochemc@gmail.com}{\textbf{briochemc}@gmail.com} \\
    California, CA, 92697\hfill \faMobile~+1 949 558 1022 \\
    USA \hfill \faSkype~briochemc



\section{Research Interests}
  A fascinating consequence of the fluid nature of open-ocean ecosystems and nutrient cycles is that biological production in one part of the ocean can influence biological production on the other side of the world.
  My research focuses on addressing fundamental scientific questions for global ocean biogeochemical cycles, using cutting-edge mathematical tools.
  My focus has been on the global biological pump, and the phosphorus, silicon, and iron cycles.
  I would like to further investigate interactions between micronutrient and macronutrient cycles using novel Green function based mathematical diagnostics.
  There is a need to represent dominant biogeochemical processes using models that are simultaneously simple, efficient, suitable for novel diagnostics, and that still retain a mechanistic behavior so that the perturbation experiments and predictions we make are correct.
  I am captivated by the potential of these techniques: by providing clear quantitative results, they help decipher complex global interactions of the oceanic nutrient cycles.


\section{Education}
\begin{entrylist}
  \entry
    {2013 - 2017}
    {Ph.D. in Applied Mathematics}
    {University of New South Wales, Sydney, Australia}
    {Supervisor: \textbf{Mark Holzer}.
    Modeling and diagnosing ocean biogeochemical cycles.\\
    \textbf{Thesis title}: \emph{The Ocean's Global Iron, Phosphorus, and Silicon Cycles: Inverse Modelling and Novel Diagnostics}.
    \begin{itemize}
        \item Global Biogeochemical Cycles, Global Biological Pump
        \item Ecosystem Modeling \& Biogenic Transport Modeling
        \item Green Functions Techniques (Path Densities, Flow Rates, Time Scales)
        \item Inverse Modeling (Newton Solver)
        \item Iron Control
        \item Nutrient Trapping
    \end{itemize}
    }

  \entry
    {2010}
    {M.Sc. in Environmental Science}
    {University of New South Wales, Sydney, Australia}
    {Study of the nature of environmental problems and the methodology of their evaluation and management.
    \begin{itemize}
        \item Geophysical Fluid Dynamics (\textbf{Mark Holzer})
        \item Project Management, Environmental Risk Management
    \end{itemize}
    }

  \entry
    {2007 - 2008}
    {M.Sc. in Finance Mathematics}
    {Paris Dauphine + ENSAE ParisTech, Paris, France}
    {MASEF (Mathematics of Insurance, Economics and Finance) Finance specialty.
    \begin{itemize}
        \item Stochastic Calculus, Levy Processes with Jumps
        \item Stochastic Differential Equations
        \item Numerical Methods (Monte Carlo)
    \end{itemize}
    }

  \entry
    {2004 - 2007}
    {M.Sc. in Mathematics \& Engineering}
    {\'{E}cole Polytechnique, Palaiseau, France}
    {Pure mathematics specialisation.
    \begin{itemize}
        \item Algebra, Arithmetics, Numerical Methods
        \item Differential Topology, Relativity
        \item Physics, Biology
    \end{itemize}
    }

  \entry
    {2001 - 2004}
    {Preparatory Classes}
    {Lyc\'{e}e Mass\'{e}na, Nice, France}
    {French Preparatory Classes, mathematics specialty.
    \begin{itemize}
        \item Linear Algebra, Topology, Numerical Methods
        \item Mechanics, Electromagnetism, Thermodynamics
    \end{itemize}
    }


\end{entrylist}

\section{Professional {\color{pblue}Exp}erience}
\begin{entrylist}
  \entry
    {Sep$\,$17 - Present}
    {Postdoctoral Researcher}
    {University of California, Irvine, CA, USA}
    {Working on improving global biogeochemistry models with J.\,Keith Moore and Fran\c{c}ois Primeau.}


  \entry
    {Mar$\,$17 - Aug$\,$17}
    {Casual Research Assistant}
    {University of New South Wales, Sydney, Australia}
    {Continuing PhD work with M.\,Holzer.
    }

  \entry
    {Jun$\,$16 - Dec$\,$16}
    {Mathematics Tutor}
    {University of New South Wales, Sydney, Australia}
    {\emph{Numerical Methods and Statistics}, 2nd year.
    }

  \entry
    {May$\,$11 - Aug$\,$12}
    {Proposal Engineer}
    {Degr\'{e}mont, Suez Environnement, Sydney, Australia}
    {Managed tendering projects for Design, Construction, Maintenance and Operation contracts.
    Participated in business development, liaising with potential clients, advertising on company capabilities.
    }

  \entry
    {Jul$\,$08 - Jun$\,$09}
    {Currency Trader Assistant}
    {Soci\'{e}t\'{e} G\'{e}n\'{e}rale Investment Banking, Paris, France}
    {MASEF Internship, researched new detection and calculation techniques for high frequency data used in automated arbitrage.
    In particular, developed real-time, unbiased stochastic moments calculation algorithms.
    }


  \entry
    {Apr$\,$07 - Jul$\,$07}
    {Mathematics Research Intern}
    {\'{E}cole Polytechnique, Palaiseau, France}
    {\'{E}cole Polytechnique Specialty (Mathematics) Internship at the Laurent Schwartz Mathematics Center under the direction of \textbf{Jean Lannes}. Calculated the Witt ring of quadratic forms defined on number fields, on the field of $p$-adic numbers, and on Dedekind rings such as the integers.
    }


  \entry
    {Sep$\,$04 - Feb$\,$05}
    {IT Intern}
    {Bioforce, Lyon, France}
    {Bioforce provides training and careers advice in aid programmes and logistics. Developed an Access database to improve communication and management.
    }

\end{entrylist}

\section{Other Skills}
\cvSkillsTwoCols{
\subsection{Programming}
\vspace{5pt}
\begin{tabular}{rl}
     {\thinfont\color{headercolor}{MATLAB}} & {Extensive use}\\
     {\thinfont\color{headercolor}{Ruby}} & {Casual use}\\
     {\thinfont\color{headercolor}{java, C++}} & {Competent}\\
     {\thinfont\color{headercolor}{Python}} & {Little experience}\\
\end{tabular}
}{
\subsection{Languages}
\vspace{5pt}
\begin{tabular}{rl}
     {\thinfont\color{headercolor}{French}} & {First language}  \\
     {\thinfont\color{headercolor}{English}} & {Fluent}\\
     {\thinfont\color{headercolor}{Italian}} & {Intermediate}  \\
     {\thinfont\color{headercolor}{Japanese}} & {Novice}\\
\end{tabular}
}


\vspace{5pt}
\section{References}
\cvRefsThreeCols{
    \cvreference
      {Mark Holzer}
      {Department of Applied Mathematics\\
       School of Mathematics and Statistics\\
       University of New South Wales\\
       NSW, 2035, Australia}
      {mholzer}
      {unsw.edu.au}
      {+61 2 9385 7109}%
  }{
    \cvreference
      {Trevor McDougall}
      {Department of Applied Mathematics\\
       School of Mathematics and Statistics\\
       University of New South Wales\\
       NSW, 2035, Australia}
      {trevor.mcdougall}
      {unsw.edu.au}
      {+61 2 9385 3498}
  }{
   % \vspace{10pt}
    \cvreference
      {Katrin Meissner}
      {Climate Change Research Centre\\
       University of New South Wales\\
       NSW, 2035, Australia}
      {k.meissner}
      {unsw.edu.au}
      {+61 2 9385 8962}
   % \cvreference
   %   {Andreas Oschlies?}
   %   {GEOMAR Helmholtz Centre for Ocean Research Kiel\\
   %    D\"{u}sternbrooker Weg 20\\
   %    D-24105 Kiel, Germany}
   %   {aoschlies}
   %   {geomar.de}
   %   {+49 431 600-1936}%
  }
\vspace{10pt}

\section{Publications}

Pasquier, B. and M.\,Holzer, \emph{Iron fertilization efficiency and the number of past and future regenerations of iron in the ocean}, Global Biogeochemical Cycles, under review.

Pasquier, B. and M.\,Holzer, \emph{Inverse-model estimates of the ocean's coupled phosphorus, silicon, and iron cycles}, Biogeosciences, 14(18), 4125--4159, \href{http://dx.doi.org/10.5194/bg-14-4125-2017}{doi: 10.5194/bg-14-4125-2017}, 2017.

Holzer, M., M.\,Frants, and B.\,Pasquier, \emph{The age of iron and iron source attribution in the ocean}, Global Biogeochem. Cycles, 30, 1454--1474, \href{http://dx.doi.org/10.1002/2016GB005418}{doi: 10.1002/2016GB005418}, 2016.

Pasquier, B. and M.\,Holzer, \emph{The plumbing of the global biological pump: Efficiency control through leaks, pathways, and time scales}, Journal of Geophysical Research: Oceans, 121, 6367--6388, \href{http://dx.doi.org/10.1002/2016JC011821}{doi: 10.1002/2016JC011821}, 2016.



%\subsection{In Press}

%\subsection{Under Review}

%\subsection{In Preparation}



\newpage

\section{Con{\color{pblue}f}erence {\color{pblue}Pres}entations}
\begin{entrylist}

  \entry{February 2018}{Ocean Sciences Meeting}{Portland, Oregon, USA}
  {Title: \emph{Inverse-model estimates of the ocean's coupled phosphorus, silicon, and iron cycles.}
  }

  \entry{February 2017}{AMOS National Conference}{Canberra, Australia}
  {Title: \emph{Response of the biological pump to perturbations in the iron supply: Global teleconnections diagnosed using an inverse model of the coupled phosphorus-silicon-iron nutrient cycles.}
  }

  \entry{February 2016}{Ocean Sciences Meeting}{New Orleans, Louisiana, USA}
  {Title: \emph{Iron control on global productivity: an efficient inverse model of the ocean's coupled phosphate, silicon, and iron cycles.}
  }

  \entry{July 2015}{AMOS National Conference}{Brisbane, Australia}
  {Title: \emph{The plumbing of the global biological pump.}
  }



\end{entrylist}

\section{Honors and Awards}
\begin{entrylist}

  \entry{2015}{Scolarship}{Cuomo Foundation, Monaco}
  {}

  \entry{2014}{Scolarship}{Fr\`{e}res Louis et Max Principale Foundation, Monaco}
  {}

    \entry{2014 - 2016}{Scolarship}{Monaco Government, Monaco}
  {Higher studies scholarship}

  \entry{2013}{Scolarship}{Monaco Government, Monaco}
  {H.S.H.~ The Prince Albert II Exceptional Scholarship}

  \entry{2013 - 2016}{Scolarship}{Monaco Scientific Centre, Monaco}
  {}

  \entry{2013 - 2016}{Tuition Fee Scholarship}{Graduate Research Shcool, UNSW, Sydney, Australia}
  {}

  \entry{2004 - 2008}{Scholarship}{Monaco Government, Monaco}{Higher studies scholarship}

\end{entrylist}

\end{document}
